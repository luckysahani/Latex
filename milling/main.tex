\documentclass[a4paper]{article}

\usepackage[english]{babel}
\usepackage[utf8x]{inputenc}
\usepackage{amsmath}
\usepackage{graphicx}
\usepackage[colorinlistoftodos]{todonotes}

\title{Milling}

\author{Lucky Sahani}

\date{\today}

\begin{document}
\maketitle



\section{How is a milling cutter mounted?}
 A milling cutter is spun about an axis while a workpiece is advanced through it in such a way that the blades of the cutter are able to shave chips of material with each pass.A horizontal mill has  the cutters  mounted on a horizontal arbor .

\section{What is the main difference between a horizontal and a vertical milling machine?}

\subsection{Horizontal Mill}
Horizontal milling machines have a spindle or cutters mounted on a horizontal arbor above an X-Y table. Some horizontal mills have a table, known as universal table, that features a rotary function for machining at different angles. Horizontal mills are optimal for machining heavier pieces because the cutters have support from the arbor, as well as a bigger cross-section area than a vertical mill. The design of the horizontal milling machine allows for the rapid removal of material off of the piece one is machining. These types of milling machines can range in size from something small enough to fit on a tabletop to room-sized machines.

\subsection{Vertical Mill}
Vertical milling machines have a spindle that moves in a vertical orientation over the table, working on the top and bottom sides of the object being machined. Vertical milling machines lend themselves to standing machinists and detailed work. These machines tend to be taller than they are wide since their operation is vertical, which works well when dealing with die sinking. Die sinking is when a cavity of a particular size and shape is machined into a steel block. The opening can then be used for molding plastic or for forging, coining, or die-casting. Below are the two types of vertical milling machines most commonly used.

\section{Question 3}

\end{document}
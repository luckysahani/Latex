%%A program to make slides on pythagoras theorem
%to include themes, animation, title slide, blocks, table and figures

\documentclass[10pt]{beamer}

\usepackage{url}			%for url's to be included in tex file 
\usepackage{hyperref}		%for url's to be included in pdf file
\usepackage{multirow}		%to be used in tables
\usepackage{graphicx}		%for graphics(here it is used to insert images)
\usepackage{wasysym}		%for using various symbols

\usetheme{Warsaw}
%default is there if not specified
\setbeamercovered{invisible}
%we can use transparent/invisible/dynamic too
\usecolortheme{beaver}
%color scheme is beaver
\setbeamertemplate{navigation symbols}{}
%remove navigation symbols

\title[Pythagroras theorem 1]{A Visual Proof Of The Pythagorean Theorem}
%Mine proof of pythagoras
\subtitle{Pythagoras theorem}
\author[Pythagoras of Samos]{Pythagoras}
%I am pythagoras
\date{\today}
%todays date

%\url{http://www.stack.nl/~jwk/latex/}
%\textcolor<2>{orange}{Example Text
%\begin{enumerate}[(I)]
%\item<1-> Point A


\begin{document}
%\maketitle              %it will show the title page again
\begin{frame}
	\titlepage			%it will show the title page just like maketitle
\end{frame}

\begin{frame}{Outline}
	\tableofcontents
\end{frame}

\section{\underline{Introduction to Pythagoras Theorem}}

\begin{frame}{Introduction}
\label{sec:intro}

%This is a introduction to my proof on pythagoras theorem  and its details....

The theorem is of fundamental importance in Euclidean Geometry where it serves 
as a basis for the definition of distance between two points.
\emph{The area of the square built upon the hypotenuse of a right
triangle is equal to the sum of the areas of the squares upon the
remaining sides}.
\end{frame}

\section{\underline{Theorem, Definition And Relation}}
\label{sec:def}
\subsection{Theorem}
\begin{frame}{Theorem}
The square of the hypotenuse (the side opposite the right angle) is equal to the sum of the squares of the other two sides
\end{frame}

\subsection{Definition}
\begin{frame}{Definition}
The theorem that the sum of the squares of the lengths of the sides of a right triangle is equal to the square of the length of the hypotenuse.
\end{frame}

\subsection{Relation}
\begin{frame}{Relation}
	\begin{itemize}
	\item For making slides
		\pause
		\visible<6-6>
		{
		\item Why not office?
		}
	\end{itemize}
In mathematics, the Pythagorean theorem or Pythagoras theorem is a relation in Euclidean geometry among the three sides of a right triangle. The theorem can be written as an equation relating the lengths of the sides a, b and c, often called the Pythagorean equation:
\begin{equation}
\label{one}
a^2+b^2=c^2
\end{equation}
where c represents the length of the hypotenuse, and a and b represent the lengths of the other two sides.

\end{frame}

\end{document}


\documentclass[]{article}

\usepackage{graphicx}
\usepackage{cite}
\usepackage{amsthm}
\usepackage[]{amsmath}
\usepackage{amssymb}

\newtheorem{thm}{Theorem}
\newtheorem{defn}{Definition}

\title{CS201 Assignment 3 }

\author{Lucky Sahani\\
      Roll - 12383\\
       Instructor - Prof. S K Mehta \\
	       Email:\texttt{luckys@iitk.ac.in}}
		\date{\today}

\begin{document}
\maketitle

\section { Question 3.1 : }
\label{sec:q1}
{\bf Question :} How many permutations, $\sigma$ , of [n] exist in which one of the following holds: $\sigma(n - 2) > (n - 2)$ or $\sigma(n - 1) > (n - 1) $ ?\\
{\bf Solution : } Let A denotes the condition : $\sigma(n - 2) > (n - 2)$ and B denote the condition  $\sigma(n - 1) > (n - 1) $ . Now , we need to find A $\cup$ B which can be calculated by the principle of inclusion and exclusion i.e.
\begin {center}
$|A \cup B| = |A| + |B| - |A \cap B| $
\end{center}
So , we have to calculate no. of permutations that satisfy A , B and $A\cap B$ :
\subsection{To calculate the number of permutations that satisfies the condition A }
Now , there are only two possibilities depending on the fact that whether $\sigma(n-2)=n-1$ or $\sigma(n-2)=n$. If $\sigma(n-2)=n-1$ is true , then number of permutations possible is same as arranging (n-1) objects among themselves , which is equal to (n-1)! . If $\sigma(n-2)=n$ is true , then number of permutations possible is same as arranging the (n-1) objects among themselves , which is equal to (n-1)! . 
\begin{center}
$\therefore$ Number of permutation that satisfy A = 2*(n - 1)!
\end{center}

\subsection{ To calculate the number of permutations that satisfies the condition B }
There is only one possibility which is $\sigma(n-1)=n$ . The total number of permutation satisfying this condition is same as permuting the remaining n-1 objects among themselves .
\begin{center}
$\therefore$ Number of permutation that satisfy B = (n - 1)!
\end{center}

\subsection{ To calculate the number of permutations that satisfies the condition  $A \cap B $ }
In this case , we have only one possibility that is $\sigma(n-1)=n$ and $\sigma(n-2)=n-1$ . Number of cases is same as permuting the remaining n-2 objects among themselves.
\begin{center}
$\therefore$ Number of permutation that satisfy $A \cap B = (n - 2)!$
\end{center}

\subsection{ To calculate the number of permutations that satisfies the condition $A \cup B $ }
By principle of inclusion and exclusion , we can say that 
\begin{center}
$|A \cup B| = |A| + |B| - |A \cap B| $
$= 2*(n-1)! + (n-1)! - (n-2)! $\\
$= 3*(n-1)! - (n-2)! $
\end{center}
Number of  permutations, $\sigma$ , of [n] , in which one of the following holds: $\sigma(n - 2) > (n - 2)$ or $\sigma(n - 1) > (n - 1) $ is equal to $ 3*(n-1)! - (n-2)! $\\
$\therefore$Number of permutations in which exactly one of the above condition holds = $|B| = 3(n - 1)! - 2(n - 2)! = (3n - 5) * (n-2)!$.



\section { Question 3.3 :}
\label{sec:q2}
{\bf Question :}
In how many permutations, $\sigma$ , $ \sigma (i) > \sigma(i + 1)$ exactly when $i \in  { 1 , 4 , 6 } ?$\\
{\bf Solution : } Following are the equations that need to be satisfied :
\begin{align}
\sigma (1) > \sigma(2)\\
\sigma (2) < \sigma(3)\\
\sigma (3) < \sigma(4)\\
\sigma (4) > \sigma(5)\\
\sigma (5) < \sigma(6)\\
\sigma (6) > \sigma(7)\\
\sigma (7) < \sigma(8)
\end{align}
Let $A_{i_1,i_2\cdots i_n}$ denote the set of permutations of [8] which have 
\begin{center}
$\sigma(i_k) > \sigma(i_k + 1)$ such that $1 \leq k \leq n.$
\end{center}
Number of permutations that satisfy the above conditions is same as number of permuataions that do satisfy $A_{1,4,6}$ and do not satisfy for i in ${ 2,3 ,5,7}$ .\\
Let $B_i = A_{1,4,6,i} $ ,$i \in \{2,3,5,7\}$.\\
Therefore , requires number of permutations is equal to number of elements not in $B_2,B_3,B_5,B_7$.\\
Applying the principle of inclusion and exclusion . We define that $|A_{i_1,i_2\cdots i_n}|$ is equal to $8!$ divided by $(k+1)!$ where k is length of consecutive indices . We can prove that by fixing 'k+1' elements in any permutations of [8] , then the number of ways we can arrange them is 1 in each permutation . Therefore , in $(K+1)!$ permutations of [8] , only 1 satisfies this constraint . Thus , number of permutations of [8] satisfying $A_{s_1,s_2,.....,s_j}$ is equal to $8! / (k+1)!$ where k is the length of all consecutive indices . 
For example $|A_{2,45}|= \frac{8!}{(3!)(2!)}$ as it has one consecutive sequence of $(4,5)$ of length 2 and one of $(2)$ of length 1.\\
$\therefore \quad$ Required number of permutations satisfying all the above conditions = \\
$|A_{1,4,6}|- \sum B_i + \sum B_i \cap B_j - \sum B_i \cap B_j \cap B_k + \sum B_i \cap B_j \cap B_k \cap B_l$\\
$Ans = |A_{1,4,6}|-|A_{1,2,4,6}|-|A_{1,3,4,6}|-|A_{1,4,5,6}|-|A_{1,4,6,7}|+|A_{1,2,3,4,6}|+|A_{1,2,4,5,6}|+|A_{1,2,4,6,7}|+|A_{1,3,4,5,6}|+|A_{1,3,4,6,7}|+|A_{1,4,5,6,7}|-|A_{1,2,3,4,5,6}|-|A_{1,2,3,4,5,7}|\\-|A_{1,2,4,5,6,7}|-|A_{1,3,4,5,6,7}|+|A_{1,2,3,4,5,6,7}|$
\paragraph{}
$= \frac{8!}{(2!)(2!)(2!)}-\frac{8!}{(3!)(2!)(2!)}-\frac{8!}{(2!)(3!)(2!)}-\frac{8!}{(2!)(4!)}-\frac{8!}{(2!)(2!)(3!)}+\frac{8!}{(5!)(2!)}+\frac{8!}{(3!)(4!)}+\frac{8!}{(3!)(2!)(3!)}+\frac{8!}{(2!)(5!)}+\frac{8!}{(2!)(3!)(3!)}+\frac{8!}{(2!)(5!)}-\frac{8!}{(7!)}-\frac{8!}{(5!)(3!)}-\frac{8!}{(3!)(5!)}-\frac{8!}{(2!)(6!)}+\frac{8!}{(8!)}$
\paragraph{}
Ans = $5040-1680-1680-840-1680+168+280+560+168+560+168-8-56-56-28+1=917$

\begin{center}
$\therefore$ Number of permutations satisfying above conditions are 917.
\end{center}


\section { Question 3.5 : }
\label{sec:q3}

{\bf Question : }How many integers less than 210 are relatively prime to 210? (We need to solve this problem using Inclusion-Exclusion principle) \\
{\bf Solution : }  The integers that are relatively prime to 210 is same as all the numbers which are not divisible by its prime factors. 
\begin{center}
Total number of integers  =210\\
$210 = 2*3*5*7$
\end{center}
Let $(A_1,A_2,A_3,A_4)$ denote the family of subset of its  factors , where , \\
$|A_1|$ denotes the numbers divisible by 2 which is less than or equal to 210 .  \\
$|A_2|$ denotes the numbers divisible by 3 which is less than or equal to 210 .  \\
$|A_3|$ denotes the numbers divisible by 5 which is less than or equal to 210 .  \\
$|A_4|$ denotes the numbers divisible by 7 which is less than or equal to 210 .  \\
Now,  by the Principle of Inclusion and Exclusion , number of integers less than 210 which are relatively prime to 210 is equal to: \\
$\Rightarrow$
(Total number of integers) - ( $|A_1 \cup A_2 \cup A_3 \cup A_4|$ ) =\\
 $   210 - ( \sum |A_i| -\sum (|A_i \cap A_j|) + \sum(|A_i \cap A_j \cap A_k|) - (|A_1 \cap A_2 \cap A_3 \cap A_4| ))\vspace{3 mm}$\\
Now,
\begin{center}
$\sum |A_i| = |A_1| + |A_2| + |A_3| + |A_4| = 105 + 70 + 42 + 30 = 247 $\\
$\sum (|A_i \cap A_j|)=   |A_1 \cap A_2| - |A_1 \cap A_3| - |A_1 \cap A_4|  - |A_2 \cap A_3| - |A_2 \cap A_4| - |A_3 \cap A_4| $
= 35 + 21 + 15 + 14 + 10 + 6 =  101 \\
$\sum (|A_i \cap A_j \cap A_j|)= |A_1 \cap A_2 \cap A_3| + |A_1 \cap A_2 \cap A_4| + |A_2 \cap A_3 \cap A_4| + |A_1 \cap A_4 \cap A_3|   $
= 7 + 5 + 2 + 3= 17 \\
\end{center}
Also , ($ |A_1 \cap A_2 \cap A_3 \cap A_4|) = 1 $ 
\begin{center}
$\therefore$ Number of integers less than 210 which are relatively prime to 210\\ = 210 - 247 +101 -17 + 1 = 48
\end{center}







\end{document}




















